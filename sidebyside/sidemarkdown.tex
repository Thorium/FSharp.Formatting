\documentclass{article}

\usepackage{xcolor}
\usepackage{graphicx}
\usepackage{fancyvrb}
\usepackage{listings}
\usepackage[T1]{fontenc}
\usepackage{hyperref}
\usepackage{amsmath}

\definecolor{officegreen}{rgb}{0, 0.5, 0}
\definecolor{navy}{rgb}{0, 0, 0.5}
\definecolor{linecolor}{rgb}{0.5, 0.6875, 0.6875}
\definecolor{outputcolor}{rgb}{0.375, 0.375, 0.375}

\newcommand{\id}[1]{\textcolor{black}{#1}}
\newcommand{\com}[1]{\textcolor{officegreen}{#1}}
\newcommand{\inact}[1]{\textcolor{gray}{#1}}
\newcommand{\kwd}[1]{\textcolor{navy}{#1}}
\newcommand{\num}[1]{\textcolor{officegreen}{#1}}
\newcommand{\ops}[1]{\textcolor{purple}{#1}}
\newcommand{\prep}[1]{\textcolor{purple}{#1}}
\newcommand{\str}[1]{\textcolor{olive}{#1}}
\newcommand{\lines}[1]{\textcolor{linecolor}{#1}}
\newcommand{\fsi}[1]{\textcolor{outputcolor}{#1}}
\newcommand{\omi}[1]{\textcolor{gray}{#1}}

% Overriding color and style of line numbers
\renewcommand{\theFancyVerbLine}{
\lines{\small \arabic{FancyVerbLine}:}}

\lstset{%
  backgroundcolor=\color{gray!15},
  basicstyle=\ttfamily,
  breaklines=true,
  columns=fullflexible
}

\title{{page-title}}
\date{}

\begin{document}

\maketitle


\section*{Example: Using Markdown Content}



This file demonstrates how to write Markdown document with
embedded F\# snippets that can be transformed into nice HTML
using the \texttt{literate.fsx} script from the \href{http://fsprojects.github.io/FSharp.Formatting}{F\# Formatting
package}.


In this case, the document itself is a valid Markdown and
you can use standard Markdown features to format the text:
\begin{itemize}
\item Here is an example of unordered list and...

\item Text formatting including \textbf{bold} and \emph{emphasis}

\end{itemize}



For more information, see the \href{http://daringfireball.net/projects/markdown}{Markdown} reference.
\subsection*{Writing F\# code}



In standard Markdown, you can include code snippets by
writing a block indented by four spaces and the code
snippet will be turned into a \texttt{<pre>} element. If you do
the same using Literate F\# tool, the code is turned into
a nicely formatted F\# snippet:
\begin{lstlisting}[numbers=left]

[escapeinside=\\\{\}]
\com{/// The Hello World of functional languages!}
\kwd{let} \kwd{rec} \lfun{factorial} \lfun{x} \ops{=} 
  \kwd{if} \lfun{x} \ops{=} \num{0} \kwd{then} \num{1} 
  \kwd{else} \lfun{x} \ops{*} {(}\lfun{factorial} {(}\lfun{x} \ops{-} \num{1}{)}{)}

\kwd{let} \id{f10} \ops{=} \lfun{factorial} \num{10}


\end{lstlisting}

\subsection*{Hiding code}



If you want to include some code in the source code,
but omit it from the output, you can use the \texttt{hide}
command. You can also use \texttt{module=...} to specify that
the snippet should be placed in a separate module
(e.g. to avoid duplicate definitions).


The value will be deffined in the F\# code that is
processed and so you can use it from other (visible)
code and get correct tool tips:
\begin{lstlisting}[numbers=left]

[escapeinside=\\\{\}]
\kwd{let} \id{answer} \ops{=} \ltyp{Hidden}{.}\id{answer}


\end{lstlisting}

\subsection*{Including other snippets}



When writing literate programs as Markdown documents,
you can also include snippets in other languages.
These will not be colorized and processed as F\#
code samples:
\begin{lstlisting}
Console.WriteLine("Hello world!");

\end{lstlisting}


This snippet is turned into a \texttt{pre} element with the
\texttt{lang} attribute set to \texttt{csharp}.


\end{document}
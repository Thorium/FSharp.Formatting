\documentclass{article}

\usepackage{xcolor}
\usepackage{graphicx}
\usepackage{fancyvrb}
\usepackage{listings}
\usepackage[T1]{fontenc}
\usepackage{hyperref}
\usepackage{amsmath}

\definecolor{officegreen}{rgb}{0, 0.5, 0}
\definecolor{navy}{rgb}{0, 0, 0.5}
\definecolor{linecolor}{rgb}{0.5, 0.6875, 0.6875}
\definecolor{outputcolor}{rgb}{0.375, 0.375, 0.375}

\newcommand{\id}[1]{\textcolor{black}{#1}}
\newcommand{\com}[1]{\textcolor{officegreen}{#1}}
\newcommand{\inact}[1]{\textcolor{gray}{#1}}
\newcommand{\kwd}[1]{\textcolor{navy}{#1}}
\newcommand{\num}[1]{\textcolor{officegreen}{#1}}
\newcommand{\ops}[1]{\textcolor{purple}{#1}}
\newcommand{\prep}[1]{\textcolor{purple}{#1}}
\newcommand{\str}[1]{\textcolor{olive}{#1}}
\newcommand{\lines}[1]{\textcolor{linecolor}{#1}}
\newcommand{\fsi}[1]{\textcolor{outputcolor}{#1}}
\newcommand{\omi}[1]{\textcolor{gray}{#1}}

% Overriding color and style of line numbers
\renewcommand{\theFancyVerbLine}{
\lines{\small \arabic{FancyVerbLine}:}}

\lstset{%
  backgroundcolor=\color{gray!15},
  basicstyle=\ttfamily,
  breaklines=true,
  columns=fullflexible
}

\title{{page-title}}
\date{}

\begin{document}

\maketitle


\section*{Literate Notebooks}



Content may be created using  \href{https://github.com/dotnet/interactive/tree/main}{.NET interactive} polyglot notebooks as the input file. Notebooks are processed by converting the notebook to a literate \texttt{.fsx} script and then passing the script through the script processing pipeline. Markdown notebook cells are passed through as comments surrounded by \texttt{(**} and \texttt{*)}, F\# code cells are passed through as code, and non-F\# code is passed through as markdown fenced code blocks between \texttt{(**} and \texttt{*)} comment markers.


The \texttt{fsdocs} tool uses \href{https://github.com/jonsequitur/dotnet-repl}{dotnet-repl} to evaluate polyglot notebooks. You need this tool to evaluate notebooks using \texttt{dotnet fsdocs [build|watch] --eval}. It can be installed into your local tool manifest using the command \texttt{dotnet tool install dotnet-repl}.


F\# Formatting tries to faithfully reproduce a notebook's native appearance when generating documents. Notebook cell outputs are passed through unchanged to preserve the notebook's html output. The below snippet demonstrates a notebook's html output for F\# records, which differs from the output you would get with the same code inside a literate scripts.
\begin{lstlisting}[numbers=left]

[escapeinside=\\\{\}]
\kwd{type} \ltyp{ContactCard} \ops{=}
    {\{} {Name}{:} \ltyp{string}
      {Phone}{:} \ltyp{string}
      {ZipCode}{:} \ltyp{string} {\}}

\com{// Create a new record}
{\{} {Name} \ops{=} \str{"Alf"}{;} {Phone} \ops{=} \str{"(555) 555-5555"}{;} {ZipCode} \ops{=} \str{"90210"} {\}}

\end{lstlisting}

<p><details open="open" class="dni-treeview"><summary><span class="dni-code-hint"><code>{ Name = &quot;Alf&quot;\n  Phone = &quot;(555) 555-5555&quot;\n  ZipCode = &quot;90210&quot; }</code></span></summary><div><table><thead><tr></tr></thead><tbody><tr><td>Name</td><td><div class="dni-plaintext"><pre>Alf</pre></div></td></tr><tr><td>Phone</td><td><div class="dni-plaintext"><pre>(555) 555-5555</pre></div></td></tr><tr><td>ZipCode</td><td><div class="dni-plaintext"><pre>90210</pre></div></td></tr></tbody></table></div></details><style>
.dni-code-hint {
    font-style: italic;
    overflow: hidden;
    white-space: nowrap;
}
.dni-treeview {
    white-space: nowrap;
}
.dni-treeview td {
    vertical-align: top;
    text-align: start;
}
details.dni-treeview {
    padding-left: 1em;
}
table td {
    text-align: start;
}
table tr { 
    vertical-align: top; 
    margin: 0em 0px;
}
table tr td pre 
{ 
    vertical-align: top !important; 
    margin: 0em 0px !important;
} 
table th {
    text-align: start;
}
</style>
</p>


\end{document}